\section{Practical - Implementing an ALU}
By the end of this practical, you will have a working understanding of:
\begin{itemize}
    \item Simple test benches in Vivado
    \item An ALU
    \item Writing Verilog code
\end{itemize}

\subsection{Scenario}
You will need to implement a Verilog module to create an 8-bit ALU with the following shown in table \ref{tbl:VSALUOP} below.
The design must also include an internal ``accumulator" register, and have a clocked input. No flags are necessary for this implementation.

\begin{table}[H]
\centering
\caption{List of operations for the VS-ALU}
\label{tbl:VSALUOP}
\begin{tabular}{|l|l|l|}
\hline
\textbf{Operation} & \textbf{Opcode} & \textbf{ALU Operation}               \\ \hline
ADD                & 0000            & Acc = A + B                          \\ \hline
SUB                & 0001            & Acc = A - B                          \\ \hline
MUL                & 0010            & Acc = A * B                          \\ \hline
DIV                & 0011            & Acc = A / B                          \\ \hline
ADDA               & 0100            & Acc = Acc + A                        \\ \hline
MULA               & 0101            & Acc = Acc * A                        \\ \hline
MAC                & 0110            & Acc = Acc + (A * B)                  \\ \hline
ROL                & 0111            & Acc = A rotated left by 1            \\ \hline
ROR                & 1000            & Acc = A rotated right by 1           \\ \hline
AND                & 1001            & Acc = A and B                        \\ \hline
OR                 & 1010            & Acc = A or B                         \\ \hline
XOR                & 1011            & Acc = A xor B                        \\ \hline
NAND               & 1100            & Acc = A nand B                       \\ \hline
ETH                & 1101             & Acc = 0xFF is A=B else 0             \\ \hline
GTH                & 1110            & Acc = 0xFF if A\textgreater{}B else 0\\ \hline
LTH                & 1111            & Acc = 0xFF if A\textless{}B else 0   \\ \hline
\end{tabular}%
\end{table}

To elaborate on the above:
\begin{itemize}
    \item You can assume A and B are unsigned
    \item \verb|ROL| and \verb|ROR| commands implement left and right shifts respectively, but the bit that is shifted out is moved back to the new bit.
    \begin{itemize}
        \item Consider \verb|1011001| as an example. with the \verb|ROL| command, the result is 0110011. With \verb|ROR| we get 1101100.
    \end{itemize}
\end{itemize}


\subsection{Deliverables}
A single group member must submit a zip file called ``Prac2\_Studnum1\_studnum2.zip". In it you must submit the following 3 files. Each file inside the compressed folder must also be correctly named with both student numbers.

\begin{enumerate}
    \item A file containing your code ``Prac2\_Studnum1\_studnum2.v"
    \item A file containing the testbench you used to test your design ``Prac2\_Studnum1\_studnum2\_tb.v"
    \item A PDF showing screen captures of your test bench output and waveform produced for the following operations:
    \begin{itemize}
        \item ADD
        \item MAC, with an ACC initial value of 0b00000101
        \item ROR
        \item LTH
    \end{itemize}
\end{enumerate}



\subsection{Marking}
Marks will be awarded for:
\begin{itemize}
    \item Code neatness (includes commenting) [5]
    \item Code completion (implementing all the ALU operations) [10]
    \item Correct results from the marking testbench [15]\\
    You are required to create your own testbench to check your operations, but when marking your module another will be used.
\end{itemize}

Marks will be deducted for not following instructions.